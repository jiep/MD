\documentclass[12pt,a4paper,twoside,openright,titlepage,final]{article}
\usepackage{fontspec}
\usepackage{amsmath}
\usepackage{amsfonts}
\usepackage{amssymb}
\usepackage{makeidx}
\usepackage{graphicx}
\usepackage[hidelinks,unicode=true]{hyperref}
\usepackage[spanish,es-nodecimaldot,es-lcroman,es-tabla,es-noshorthands]{babel}
\usepackage[left=3cm,right=2cm, bottom=4cm]{geometry}
\usepackage{natbib}
\usepackage{microtype}
\usepackage{ifdraft}
\usepackage{verbatim}
\usepackage[nottoc]{tocbibind}
\usepackage{pdflscape}
\usepackage[obeyDraft]{todonotes}
\ifdraft{
	\usepackage{draftwatermark}
	\SetWatermarkText{BORRADOR}
	\SetWatermarkScale{0.7}
	\SetWatermarkColor{red}
}{}
\usepackage{booktabs}
\usepackage{longtable}
\usepackage{calc}
\usepackage{array}
\usepackage{caption}
\usepackage{subfigure}
\usepackage{footnote}
\usepackage{url}
\usepackage[titletoc]{appendix}

\setsansfont[Ligatures=TeX]{texgyreadventor}
\setmainfont[Ligatures=TeX]{texgyrepagella}
\setmonofont{FreeMono}

\usetikzlibrary{decorations.pathreplacing}

\input{portada}

\author{José Ignacio Escribano}

\title{}

\setlength{\parindent}{0pt}
\usepackage{fancyvrb}

\begin{document}

\pagenumbering{alph}
\setcounter{page}{1}

\portada{Caso Práctico I}{Minería de datos}{Componentes principales y análisis de correspondencias}{José Ignacio Escribano}{Móstoles}

\listoffigures
\thispagestyle{empty}
\newpage

\listoftables
\thispagestyle{empty}
\newpage

\tableofcontents
\thispagestyle{empty}
\newpage


\pagenumbering{arabic}
\setcounter{page}{1}

\section{Introducción}

En este caso práctico, utilizaremos distintas bases de datos para poner en práctica lo aprendido sobre componentes principales y análisis de correspondencias.\\

En la primera cuestión de evaluación relizaremos un análisis de componentes principales y en la segunda un análisis de correspondecias. En ambos casos utilizaremos el software estadístico R.

\section{Resolución de las cuestiones de evaluación}

\subsection{Primera cuestión}

En esta primera cuestión, debemos realizar un análisis de componentes principales con los datos gabriel1971 del paquete bpca de R. En él se recogen las comodidades de hogares en varias zonas de Jerusalén.\\

Para comenzar con el PCA, observamos las variables y los datos (por su pequeño tamaño) de los que consta esta base de datos:

\begin{Verbatim}[fontsize=\scriptsize]
             CRISTIAN ARMENIAN JEWISH MOSLEM MODERN.1 MODERN.2 OTHER.1 OTHER.2  RUR
toilet           98.2     97.2   97.3   96.9     97.6     94.4    90.2    94.0 70.5
kitchen          78.8     81.0   65.6   73.3     91.4     88.7    82.2    84.2 55.1
bath             14.4     17.6    6.0    9.6     56.2     69.5    31.8    19.5 10.7
eletricity       86.2     82.1   54.5   74.7     87.2     80.4    68.6    65.5 26.1
water            32.9     30.3   21.1   26.9     80.1     74.3    46.3    36.2  9.8
radio            73.0     70.4   53.0   60.5     81.2     78.0    67.9    64.8 57.1
tv set            4.6      6.0    1.5    3.4     12.7     23.0     5.6     2.7  1.3
refrigerator     29.2     26.3    5.3   10.5     52.8     49.7    21.7     9.5  1.2
\end{Verbatim}

Tenemos 8 variables en esta base de datos: CRISTIAN (cristianos), ARMENIAN (armenios), JEWISH (judíos), MOSLEM (musulmanes), MODERN.1 (modernos.1), MODERN.2 (modernos.2), OTHER.1 (otros.1), OTHER.2 (otros.2) y RUR (población rural).\\

Las comodidades de los hogares son lavabo, cocina, baño, electricidad, agua, radio, televisión y nevera.\\

Antes de realizar el PCA, realizamos un resumen de cada variable para ver si existen grandes diferencias de escala entre las variables.

\begin{Verbatim}[fontsize=\scriptsize]
   CRISTIAN        ARMENIAN         JEWISH           MOSLEM         MODERN.1       
 Min.   : 4.60   Min.   : 6.00   Min.   : 1.500   Min.   : 3.40   Min.   :12.70  
 1st Qu.:25.50   1st Qu.:24.12   1st Qu.: 5.825   1st Qu.:10.28   1st Qu.:55.35   
 Median :52.95   Median :50.35   Median :37.050   Median :43.70   Median :80.65   
 Mean   :52.16   Mean   :51.36   Mean   :38.038   Mean   :44.48   Mean   :69.90   
 3rd Qu.:80.65   3rd Qu.:81.28   3rd Qu.:57.275   3rd Qu.:73.65   3rd Qu.:88.25   
 Max.   :98.20   Max.   :97.20   Max.   :97.300   Max.   :96.90   Max.   :97.60  
 
    MODERN.2        OTHER.1         OTHER.2           RUR        
 Min.   :23.00   Min.   : 5.60   Min.   : 2.70   Min.   : 1.200  
 1st Qu.:64.55   1st Qu.:29.27   1st Qu.:17.00   1st Qu.: 7.675  
 Median :76.15   Median :57.10   Median :50.50   Median :18.400  
 Mean   :69.75   Mean   :51.79   Mean   :47.05   Mean   :28.975  
 3rd Qu.:82.47   3rd Qu.:72.00   3rd Qu.:70.17   3rd Qu.:55.600  
 Max.   :94.40   Max.   :90.20   Max.   :94.00   Max.   :70.500  
\end{Verbatim}

Puesto que todas las variables son porcentajes, no será necesario escalar las variables.\\

Calculamos los autovalores de la matriz de covarianzas de cada componente principal. Tenemos lo siguiente:

\begin{Verbatim}[fontsize=\scriptsize]
Standard deviations:
[1] 9.320779e+01 1.832205e+01 1.270250e+01 6.285602e+00 4.360009e+00 3.173772e+00 1.334664e+00 4.637537e-15

Rotation:
                PC1          PC2         PC3         PC4         PC5         PC6        PC7         PC8
CRISTIAN -0.3774826  0.121218290  0.48084436  0.40488521  0.09488569  0.04718208  0.1684884  0.61020121
ARMENIAN -0.3705463  0.139480696  0.33349190  0.29316420 -0.12562056 -0.26551516 -0.5595880 -0.47868586
JEWISH   -0.3661639  0.283772155 -0.14629893 -0.48523653  0.63125444  0.05070960 -0.3147304  0.15768194
MOSLEM   -0.3830485  0.206214896  0.21868882 -0.33508268 -0.07215394 -0.15428314  0.6858741 -0.38841777
MODERN.1 -0.2712244 -0.650316615  0.08542964  0.11706375  0.31135408  0.53403172  0.0548636 -0.30654238
MODERN.2 -0.2201737 -0.555023062 -0.20049950 -0.05554059  0.08031210 -0.72925238  0.0137473  0.14695975
OTHER.1  -0.3214890 -0.189304742 -0.14718330 -0.06899714 -0.33431892  0.08981627  0.0167303  0.28706773
OTHER.2  -0.3710368  0.007994056 -0.20526225 -0.28162871 -0.58680361  0.27449421 -0.2179725  0.08803244
RUR      -0.2762924  0.275567693 -0.69068799  0.54911184  0.10967589  0.02208732  0.1949449 -0.13061157
\end{Verbatim}

Tenemos que las dos primeras componentes principales acumulan más del 95\% de la varianza de los datos.\\

Podemos ver estos datos de forma más gráfica en la Figura~\ref{fig:varianza_pca}.\\

\begin{figure}[tbph!]
\centering
\includegraphics[width=0.8\linewidth]{imagenes/varianza_pca}
\caption{Importancia de cada componente}
\label{fig:varianza_pca}
\end{figure}

Existe una gran diferencia entre la primera y las otras componentes. De forma numérica se obtiene lo siguiente:

\begin{Verbatim}[fontsize=\scriptsize]
Importance of components:
                           PC1      PC2      PC3     PC4     PC5     PC6     PC7       PC8
Standard deviation     93.2078 18.32205 12.70250 6.28560 4.36001 3.17377 1.33466 4.638e-15
Proportion of Variance  0.9387  0.03627  0.01743 0.00427 0.00205 0.00109 0.00019 0.000e+00
Cumulative Proportion   0.9387  0.97496  0.99240 0.99667 0.99872 0.99981 1.00000 1.000e+00
\end{Verbatim}

La primera componentes explica más del 93\% de la varianza de los datos, y entre las dos primeras casi el 97.5\% de la varianza.\\

Cogeremos las dos primeras componentes para representar los datos en un diagrama de dispersión (Figura~\ref{fig:dos_componentes}).\\

\begin{figure}[tbph!]
\centering
\includegraphics[width=0.8\linewidth]{imagenes/dos_componentes}
\caption{Proyección sobre las dos componentes principales}
\label{fig:dos_componentes}
\end{figure}

Este gráfico tiene difícil interpretabilidad, por lo que eliminamos los puntos y ponemos el nombre de cada comodidad (televisión, electricidad, baño, etc) como se puede ver en la Figura~\ref{fig:dos_componentes_nombres}.\\

\begin{figure}[tbph!]
\centering
\includegraphics[width=0.8\linewidth]{imagenes/dos_componentes_nombres}
\caption{Proyección sobre las dos componentes principales con el nombre de las comodidades}
\label{fig:dos_componentes_nombres}
\end{figure}

Se puede observar un comportamiento similar entre la cocina, electricidad y radio, y entre agua, baño y nevera, ya que se encuentran muy próximos entre sí en el gráfico. En el primer caso, el uso de esas comodidades del hogar es bastante extendido entre las distintas zonas de Jerusalén, mientras que en el segundo caso, estas comodidades son frenos frecuentes entre las distintas zonas de Jerusalén.\\
Asimismo, se observan dos casos extremos: por un lado la televisión, y por el otro, el lavabo. En el primero, su presencia entre las distintas zonas de Jerusalén es testimonial, mientras que en el segundo es ampliamente utilizado en todas las zonas de Jerusalén.\\

Podemos mejorar el gráfico utilizando el biplot como se puede ver en la Figura~\ref{fig:biplot}.\\

\begin{figure}[tbph!]
\centering
\includegraphics[width=0.5\linewidth]{imagenes/biplot}
\caption{Biplot}
\label{fig:biplot}
\end{figure}

Se observa que las flechas con las zonas de Jerusalén apuntan hacia las comodidades más extendidas entre las que se encuentran el lavabo, la radio, la electricidad y la cocina, como habíamos visto anteriormente.\\

\subsection{Segunda cuestión}

Comenzamos viendo los datos de los que disponenemos:

\begin{verbatim}
, , Sex = Male

       Eye
Hair    Brown Blue Hazel Green
  Black    32   11    10     3
  Brown    53   50    25    15
  Red      10   10     7     7
  Blond     3   30     5     8

, , Sex = Female

       Eye
Hair    Brown Blue Hazel Green
  Black    36    9     5     2
  Brown    66   34    29    14
  Red      16    7     7     7
  Blond     4   64     5     8

\end{verbatim}

Los datos se encuentran dividos por sexo, por lo que debemos agrupar los datos de cada sexo para obtener los datos globales de hombres y mujeres.\\

A continuación realizamos tres análisis de correspondencias: para mujeres, hombres y datos conjuntos.\\

Realizamos, en primer lugar, el análisis para las mujeres. La Figura~\ref{fig:mujeres_ca} muestra el resultado gráfico del análisis de correspondencias.\\

\begin{figure}[tbph!]
\centering
\includegraphics[width=\linewidth]{imagenes/mujeres_ca}
\caption{Análisis de correspondencias para las mujeres}
\label{fig:mujeres_ca}
\end{figure}

Se puede observar la cercanía entre el pelo negro y ojos marrones, lo que hace suponer la aparición frecuente de estos colores de pelo y ojos entre las mujeres. De igual forma, se observa la cercanía entre entre el pelo rubio y ojos azules lo que supone la alta presencia de estos colores de pelo y ojos. También, hay mucha cercanía entre el color de pelo rojo y ojos avellana (hazel), lo que hace pensar en la presencia de muchas mujeres con estos colores.\\

Por último, notar que el color de pelo marrón y el color de ojos verdes parecen no estar cerca de ningún otro color. El primero, se encuentra en el centro del gráfico, por lo que hace pensar en que aparece tener características especiales. El segundo, también parece estar alejado de todos los demás, lo que hace pensar que el color de ojos verdes se da con distintos colores de pelo.\\

\begin{figure}[tbph!]
\centering
\includegraphics[width=\linewidth]{imagenes/hombres_ca}
\caption{Análisis de correspondencias para las hombres}
\label{fig:hombres_ca}
\end{figure}

En los hombres (Figura~\ref{fig:hombres_ca}), se vuelven a dar los casos entre el color de ojos marrones y el pelo negro, y el color de ojos azul y  el pelo rubio. También se observa gran cercanía entre el color rojo de pelo y el verde de ojo, lo que pone de manifiesto la relación entre estos dos colores en los hombres.\\

En el centro del gráfico se sitúa el color de pelo marrón y el color de ojos avellana, lo que parece indicar que estos colores no tienen patrones determinados en los hombres.\\ 

\begin{figure}[tbph!]
\centering
\includegraphics[width=\linewidth]{imagenes/global_ca}
\caption{Análisis de correspondencias para los datos conjuntos}
\label{fig:global_ca}
\end{figure}

En los datos conjuntos (Figura~\ref{fig:global_ca}) se observa la misma tendencia en los casos anteriores: la cercanía entre los colores de pelo y ojos claros y oscuros. En la parte izquierda del gráfico se muestran los oscuros y en la derecha los claros. Entre todos los demás colores de pelo y ojos no parece haber especial cercanía entre ellos, lo que parece indicar la indiferencia de color y pelo entre ellos.\\

En definitiva, se observa una clara tendencia a aparecer los colores claros u oscuros tanto en pelo como ojos (pelo negro con ojos marrones y pelo rubio con ojos azules). En todos los demás casos no parece haber un patrón claro, aunque en el caso de las mujeres se da la cercanía entre el pelo rojo y los ojos avellana. 

\section{Conclusiones}

En este caso práctico hemos visto cómo realizar análisis de componentes principales y de correspondencias. En ambos casos hemos utilizado R, junto con librerías que implementan métodos para realizar tales análisis. En el primer caso, hemos utilizado distintos tipos de gráficos para analizar las comodidades del hogar entre distintas zonas de Jerusalén, entre los que se encuentra el biplot. En el segundo caso, hemos realizado un análisis de correspondencia para establecer relaciones entre color de pelo y ojos entre hombres y mujeres. 

\clearpage

\section{Código R}

\verbatiminput{caso_1.R}

\end{document}